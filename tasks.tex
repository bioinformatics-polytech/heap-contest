\documentclass[a4]{article}
\pagestyle{myheadings}
\setlength{\parindent}{5ex}
%%%%%%%%%%%%%%%%%%%
% Packages/Macros %
%%%%%%%%%%%%%%%%%%%
\usepackage{mathrsfs}


\usepackage{fancyhdr}
\pagestyle{fancy}
\lhead{}
\chead{}
\rhead{}
\lfoot{}
\cfoot{} 
\rfoot{\normalsize\thepage}
\renewcommand{\headrulewidth}{0pt}
\renewcommand{\footrulewidth}{0pt}
\newcommand{\RomanNumeralCaps}[1]
    {\MakeUppercase{\romannumeral #1}}

\usepackage{amssymb,latexsym}  % Standard packages
\usepackage[utf8]{inputenc}
\usepackage[russian]{babel}
\usepackage{MnSymbol}
\usepackage{mathrsfs}
\usepackage{amsmath,amsthm}
\usepackage{indentfirst}
\usepackage{graphicx}%,vmargin}
\usepackage{graphicx}
\graphicspath{{pictures/}} 
\usepackage{verbatim}
\usepackage{color}
\usepackage{color,colortbl}
\usepackage[nottoc,numbib]{tocbibind}
\usepackage{float}
\usepackage{multirow}
\usepackage{hhline}

\usepackage{listings}
\definecolor{codegreen}{rgb}{0,0.6,0}
\definecolor{codegray}{rgb}{1,1,1}
\definecolor{codepurple}{rgb}{0.58,0,0.82}
\definecolor{backcolour}{rgb}{0.95,0.95,0.92}
 
\lstdefinestyle{mystyle}{
    backgroundcolor=\color{backcolour},   
    commentstyle=\color{codegreen},
    keywordstyle=\color{magenta},
    numberstyle=\tiny\color{codegray},
    stringstyle=\color{codepurple},
    basicstyle=\footnotesize,
    breakatwhitespace=false,         
    breaklines=true,                 
    captionpos=b,                    
    keepspaces=true,                 
    numbers=left,                    
    numbersep=5pt,                  
    showspaces=false,                
    showstringspaces=false,
    showtabs=false,                  
    tabsize=2
}
 
\lstset{style=mystyle}

\usepackage{url}
\urldef\myurl\url{foo%.com}
\def\UrlBreaks{\do\/\do-}
\usepackage{breakurl}
\Urlmuskip=0mu plus 1mu



\DeclareGraphicsExtensions{.pdf,.png,.jpg}% -- настройка картинок

\usepackage{epigraph} %%% to make inspirational quotes.
\usepackage[all]{xy} %for XyPic'a
\usepackage{color} 
\usepackage{amscd} %для коммутативных диграмм
%\usepackage[colorlinks,urlcolor=red]{hyperref}

%\renewcommand{\baselinestretch}{1.5}
%\sloppy
%\usepackage{listings}
%\lstset{numbers=left}
%\setmarginsrb{2cm}{1.5cm}{1cm}{1.5cm}{0pt}{0mm}{0pt}{13mm}


\newtheorem{Lemma}{Лемма}[section]
\newtheorem{Proposition}{Предложение}[section]
\newtheorem{Theorem}{Теорема}[section]
\newtheorem{Corollary}{Следствие}[section]
\newtheorem{Remark}{Замечание}[section]
\newtheorem{Definition}{Определение}[section]
\newtheorem{Designations}{Обозначение}[section]




%%%%%%%%%%%%%%%%%%%%%%% 
%Подготовка оглавления% 
%%%%%%%%%%%%%%%%%%%%%%% 
\usepackage[titles]{tocloft}
\renewcommand{\cftdotsep}{2} %частота точек
\renewcommand\cftsecleader{\cftdotfill{\cftdotsep}}
\renewcommand{\cfttoctitlefont}{\hspace{0.38\textwidth} \LARGE\bfseries} 
\renewcommand{\cftsecaftersnum}{.}
\renewcommand{\cftsubsecaftersnum}{.}
\renewcommand{\cftbeforetoctitleskip}{-1em} 
\renewcommand{\cftaftertoctitle}{\mbox{}\hfill \\ \mbox{}\hfill{\footnotesize Стр.}\vspace{-0.5em}} 
%\renewcommand{\cftchapfont}{\normalsize\bfseries \MakeUppercase{\chaptername} } 
%\renewcommand{\cftsecfont}{\hspace{1pt}} 
\renewcommand{\cftsubsecfont}{\hspace{1pt}} 
%\renewcommand{\cftbeforechapskip}{1em} 
\renewcommand{\cftparskip}{3mm} %определяет величину отступа в оглавлении
\setcounter{tocdepth}{5} 
\renewcommand{\listoffigures}{\begingroup %добавляем номер в список иллюстраций
\tocsection
\tocfile{\listfigurename}{lof}
\endgroup}
\renewcommand{\listoftables}{\begingroup %добавляем номер в список иллюстраций
\tocsection
\tocfile{\listtablename}{lot}
\endgroup}


%\renewcommand{\thelikesection}{(\roman{likesection})}
%%%%%%%%%%%
% Margins %
%%%%%%%%%%%
\addtolength{\textwidth}{0.7in}
\textheight=630pt
\addtolength{\evensidemargin}{-0.4in}
\addtolength{\oddsidemargin}{-0.4in}
\addtolength{\topmargin}{-0.4in}

%%%%%%%%%%%%%%%%%%%%%%%%%%%%%%%%%%%
%%%%%%Переопределение chapter%%%%%% 
%%%%%%%%%%%%%%%%%%%%%%%%%%%%%%%%%%%
\newcommand{\empline}{\mbox{}\newline} 
\newcommand{\likechapterheading}[1]{ 
\begin{center} 
\textbf{\MakeUppercase{#1}} 
\end{center} 
\empline} 

%%%%%%%Запиливание переопределённого chapter в оглавление%%%%%% 
\makeatletter 
\renewcommand{\@dotsep}{2} 
\newcommand{\l@likechapter}[2]{{\bfseries\@dottedtocline{0}{0pt}{0pt}{#1}{#2}}} 
\makeatother 
\newcommand{\likechapter}[1]{ 
\likechapterheading{#1} 
\addcontentsline{toc}{likechapter}{\MakeUppercase{#1}}} 




\usepackage{xcolor}
\usepackage{hyperref}
\definecolor{linkcolor}{HTML}{000000} % цвет ссылок
\definecolor{urlcolor}{HTML}{AA1622} % цвет гиперссылок
 
\hypersetup{pdfstartview=FitH,  linkcolor=linkcolor,urlcolor=urlcolor, colorlinks=true}

%%%%%%%%%%%%
% Document %
%%%%%%%%%%%%

%%%%%%%%%%%%%%%%%%%%%%%%%%%%%
%%%%%%главы -- section*%%%%%%
%%%%section -- subsection%%%%
%subsection -- subsubsection%
%%%%%%%%%%%%%%%%%%%%%%%%%%%%%
\def \newstr {\medskip \par \noindent} 
\begin{document}



\section*{Задача 1}
\label{sec:orgb62fe60}
\subsection*{Постановка}
\label{sec:org37954e9}
У пещерного человека есть \(n\) камней. Он выбирает два самых тяжелых камня и бьет ими друг о друга. Предположим что веса камней \(x\) и \(y\).
\begin{itemize}
    \item Если \(x\)=\(y\), то оба камня разбиваются.
    \item Если \(x\)<\(y\), то камень с весом \(x\) разбивается, а вес второго камня будет равным \(y\)-\(x\).
\end{itemize}
Пещерный человек продолжает это делать пока останется не более одного камня. \\
Найдите минимальный возможный вес оставшегося камня. Если камней не осталось выведите 0.
\subsection*{Входные данные}
\label{sec:orgc51833b}
Cписок весов камней, длины \(n\).

\subsection*{Выходные данные}
\label{sec:org91cd1c2}
Выведите вес оставшегося камня.

\subsection*{Пример 1}
\label{sec:org1b720b0}

\begin{table}[H]
\begin{center}
\begin{tabular}{|m{4cm}|m{4cm}|}
\hline
Входные данные & Выходные данные \\ \hline
2 7 4 1 8 1
&
1
\\ \hline
\end{tabular}
\end{center}
\end{table}

\subsection*{Пример 2}
\label{sec:org2aeecb4}

\begin{table}[H]
\begin{center}
\begin{tabular}{|m{4cm}|m{4cm}|}
\hline
Входные данные & Выходные данные \\ \hline
2 7 4 1 8
&
0
\\ \hline
\end{tabular}
\end{center}
\end{table}

\pagebreak
\section*{Задача 2}
\label{sec:orgef181bd}
\subsection*{Постановка}
\label{sec:orgad8a20e}
Роботу передают команды в виде последовательности букв, где каждая буква представляет отдельную команду. Робот может выполнять команды в любом порядке. Каждая команда выполняется за одну секунду. В каждую отдельную секунду робот может или выполнять команду, или бездействовать. \\
Однако, в связи с функциональными особенносями робота, он не может выполнять одну и ту же команду в течение \(n\) секунд, то есть между любыми двумя одинаковыми командами должно быть не менее \(n\) секунд.
Вычислите время на за которое робот выполнит все команды.

\subsection*{Входные данные}
\label{sec:orgc51833b}
В первой строке записана последовательность команд, состоящая из символов a-z.\\
Во второй строке записано \(n\)- время востановления в секундах
\subsection*{Выходные данные}
\label{sec:orgf9da829}
Выведите количество секунд необходимых для выполнения всех команд.

\subsection*{Пример 1}
\label{sec:orgd7d348d}

\begin{table}[H]
\begin{center}
\begin{tabular}{|m{4cm}|m{4cm}|}
\hline
Входные данные & Выходные данные \\ \hline
a a a b b b 

0
&
6
\\ \hline
\end{tabular}
\end{center}
\end{table}
\subsection*{Пример 2}
\label{sec:orgd7d348d}

\begin{table}[H]
\begin{center}
\begin{tabular}{|m{4cm}|m{4cm}|}
\hline
Входные данные & Выходные данные \\ \hline
a a a b b b 

2
&
8
\\ \hline
\end{tabular}
\end{center}
\end{table}
\pagebreak
\section*{Задача 3}
\label{sec:org570b899}
\subsection*{Постановка}
\label{sec:orga2b5149}
Число будет являться "привлекательным" если оно положительное и его простые множители находятся в массиве primeNumders.
Найдите \(n\)-ое "привелкательное" число.
\subsection*{Входные данные}
\label{sec:orgeb4908d}
В первой строке записано \(n\) - номер "привлекательного" необходимого привлекательного числа.
\label{sec:orged795e8}
Во второй строке записан массив primeNumders.

\subsection*{Выходные данные}
\label{sec:orged795e8}
Выведите \(n\)-ое "привлекательное" число.
\subsection*{Пример 1}
\label{sec:org6a26c04}

\begin{table}[H]
\begin{center}
\begin{tabular}{|m{4cm}|m{4cm}|}
\hline
Входные данные & Выходные данные \\ \hline
1

2 3 5
&
1 
\\ \hline
\end{tabular}
\end{center}
\end{table}

\subsection*{Пример 2}
\label{sec:orge96f7c4}

\begin{table}[H]
\begin{center}
\begin{tabular}{|m{4cm}|m{4cm}|}
\hline
Входные данные & Выходные данные \\ \hline
12 

2 7 13 19
&
32
\\ \hline
\end{tabular}
\end{center}
\end{table}

\pagebreak
\section*{Задача 4}
\label{sec:orgb1f46a6}
\subsection*{Постановка}
\label{sec:orge854c50}
В стране Турляндии \(n\) озер. Если над озером идет дождь оно становится переполненным. Если над переполненным озером пойдет дождь то будет наводнение. Перед сезоном дождей синоптики подготовили идеальный прогноз, предсказывающий в какие дни будут дожди. Помогите правительству Турляндии избежать наводнений. В день когда дождя не будет, на одном из озер можно установить дренаж для уменьшения излишек воды. Если установить дренаж на непереполненное озеро то ничего не произойдет.

\subsection*{Входные данные}
\label{sec:orge854c50}
Введите прогноз синоптиков. Прогноз представляет собой массив целочисленных значений.
\begin{itemize}
    \item Если forecast[i]=0, то дождя в этот день не будет и можно поставить дренаж на одно из озер.
    \item Если forecast[i]>0, то будет дождь и озеро с номером forecast[i] будет переполнено.
 \end{itemize}
\subsection*{Выходные данные}
\label{sec:org1ab7414}
Выведите массив отражающий действия правительства Турляндии.
\begin{itemize}
    \item len(ans) = len(forecast).
    \item ans[i] = -1, если в i-й день идет дождь.
    \item ans[i] = \(k\), где \(k\) номер озера на котором поставят дренаж в день без дождя.
 \end{itemize}
 Если не удается избежать наводнения то верните пустой массив.
\subsection*{Пример 1}
\label{sec:org25482f8}

\begin{table}[H]
\begin{center}
\begin{tabular}{|m{4cm}|m{4cm}|}
\hline
Входные данные & Выходные данные \\ \hline
1 2 3 4
&
-1 -1 -1 -1
\\ \hline
\end{tabular}
\end{center}
\end{table}
\subsection*{Пример 2}
\begin{table}[H]
\begin{center}
\begin{tabular}{|m{4cm}|m{4cm}|}
\hline
Входные данные & Выходные данные \\ \hline
1 2 0 0 2 1
&
-1 -1 2 1 -1 -1
\\ \hline
\end{tabular}
\end{center}
\end{table}
\subsection*{Пример 3}
\begin{table}[H]
\begin{center}
\begin{tabular}{|m{4cm}|m{4cm}|}
\hline
Входные данные & Выходные данные \\ \hline
1 2 0 1 2
&
Null
\\ \hline
\end{tabular}
\end{center}
\end{table}
\pagebreak

\section*{Задача 5}
\label{sec:orgb1f46a6}
\subsection*{Постановка}
\label{sec:orge854c50}
В городе Паралелоград построили аэропорт. Для нормального функционирования которого необходимо просчитать силуэт города. Силуэт города - контур, образованный всеми зданиями города, если смотреть с растояния. Все здания представляют собой идеальные прямоугольники, основанные на абсолютно плоской поверхности на высоте 0. Помогите руководству аэропорта просчитать силуэт города.
\subsection*{Входные данные}
\label{sec:orge854c50}
Геометрическая информация о каждом здании дана в массиве зданий, где здание описывается тремя параметрами [\(x\),\(y\),\(h\)]:
\begin{itemize}
    \item x - координата левого края здания.
    \item y - координата правого края здания.
    \item h - высота здания.
 \end{itemize}
\subsection*{Выходные данные}
\label{sec:org1ab7414}
Выведите силуэт города в виде «ключевых точек», отсортированных по их координате x в форме [[x1, y1], [x2, y2], ...]. Каждая ключевая точка является левой конечной точкой некоторого горизонтального сегмента на линии силуэта, за исключением последней точки в списке, которая всегда имеет координату y=0 и используется для обозначения окончания линии силуэта там, где заканчивается крайнее правое здание.
\subsection*{Пример 1}
\label{sec:org25482f8}

\begin{table}[H]
\begin{center}
\begin{tabular}{|m{4cm}|m{4cm}|}
\hline
Входные данные & Выходные данные \\ \hline
[[0,2,3],[2,5,3]]
&
[[0,3],[5,0]]
\\ \hline
\end{tabular}
\end{center}
\end{table}
\subsection*{Пример 2}
\begin{table}[H]
\begin{center}
\begin{tabular}{|m{4cm}|m{4cm}|}
\hline
Входные данные & Выходные данные \\ \hline
[[2,9,10],[3,7,15],[5,12,12]

,[15,20,10],[19,24,8]]
&
[[2,10],[3,15],[7,12],

[12,0],[15,10],[20,8],[24,0]]
\\ \hline
\end{tabular}
\end{center}
\end{table}

\end{document}